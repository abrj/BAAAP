\section{Client}

It was a requirement for this project that we too made a client that uses the WCF services. 
For this task we chose to use Microsoft's ASP.NET Framework together with their MVC4 framework
that is suitable for web applications. We chose to use MVC4 and thereby the Model, View, Controller
design pattern, due to the fact that none from our group had really done any web development in
ASP.NET before, but one did have some basic experience with the MVC design pattern from some 
Ruby on Rails projects. The reason that we did not choose to use another framework than ASP.NET
was simply that we thought it was mandatory to use a Microsoft technology. 
We did, however, not use the MVC4 Framework exactly how it is supposed to be used. Where our
approach differentiated from the intended way to use it, was that we did not use the auto generated
models in the Framework. This was due to the fact that we are not running the server and client
the same place like one would usually do when using the MVC4 Framework for web development. 
Instead we are running the models and database on our WCF server.
Together with MVC4 we basically only used HTML5 generated by the Razor syntax. For the 
basic layout we used CSS generated by Twitters Bootstrap framework. We have only used very little
Javascript that has been auto generated by Bootstrap.

We have almost entirely used the SOAP endpoint from the WCF service for the client. The HTML5 
video player does however not support stream objects, which is what our SOAP endpoint provides
for our video streaming method. For this we used a RESTful call instead that provides a
URL that the HTML5 video player supports.

We do unfortunately have one method that we have been unable to get to work. The method that
does not work is the upload method. It works locally on the server, but as soon as we try to
upload remotely it fails. We have not been able to solve this for the client, but the 
client still works fine as a proof of concept. Other than that we do also have a few other
methods that are not implemented in our own client. The explanation to this is that our
collaborating group from Singapore wanted the functionality in question.

The styling of the client was one of the most challenging things to do, since none of the 
group members have ever really written either HTML5 or CSS in a considerable scale. The person
with the main responsibility for the client did however find it very exciting to do some web 
programming.

We have aimed to make our client as intuitive and easy to use as possible. This has been done
by using the before mentioned Bootstrap framework, and by trying to recreate some of the basic
layouts known from some of the known web sites such as Facebook and Twitter, to make sure
that the user finds what he is looking for with ease.



