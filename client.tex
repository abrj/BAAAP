\section{The ITU RentItClient}
It is a requirement for this project that we too made a client that uses the WCF services. 
For this task we chose to use Microsoft's ASP.NET Framework together with their MVC4 framework
that is suitable for web applications. We chose to use MVC4 and thereby the Model, View, Controller
design pattern, due to the many built-in functionalities and experience with the MVC design pattern from former 
Ruby on Rails projects. It also fit with trying to combine many Microsoft technologies, something we felt the course led us on to.

It should be noted, that we do not utilize the full range of functionality in the MVC4 framework.
Where our approach differentiates from the intended way to use it, is that we do not use the auto
generated models in the Framework. This is due to the fact that we are not running the server and
client the same place, like one would usually do when using the MVC4 Framework for web development.
Instead we are running the models and database on our RentIt server. Together with MVC4 we used HTML5
generated by the Razor syntax, and for the basic layout we used CSS generated by 
Twitters Bootstrap framework. We have also used small pieces of Javascript that has been auto generated by Bootstrap. \\

We have almost entirely used the SOAP endpoint from the WCF service for the client. The HTML5  video player does however not support stream objects, which is what our SOAP endpoint provides
for video streaming. For this we used a RESTful call instead that provides a
URL for the HTML5 video player, which it supports. More on this has been noted in section \ref{VideoStreaming}. The functionality of our client, mostly reflects our WCF sevice interface, with a few exceptions. Some of the methods were required by our collaborating SMU group, and have therefore not been implemented in our own client.

The styling of the client was written with a combination of HTML5 or CSS, technologies we dont have expirience with in a considerable scale. We have aimed to make our client as intuitive and easy to use as possible. This has been done
by using the before mentioned Bootstrap framework, and by trying to recreate some of the basic
layouts known from some of the known web sites such as Facebook and Twitter, to make sure
that the user finds what he is looking for with ease.



