\part{Final Thoughts}
In this part of the report we will present the reader with the final thoughts about the project and whether or not we have fulfilled all requirements from section \ref{Requirements} and the relation to the use-cases.

\section{Future Perspective}
For this section we will describe the features and functionality that we would like to have implemented in the system, but for different reasons didn't do.\\


\textbf{Advertisements}\\

As mentioned in section \ref{BusinessModel} the system would be financed through advertisements. The original idea was to start every rented video with a short commercial from Google Ad's. This would generate the needed financial base for being able to pay the uploaders of the tutorials and at the same time generate some profit for the system. We have not implemented this due to an extensive approval process at Google and the overhead of time it would take to implement it in a way that would be approved by Google. The approval requires two steps. The first is signing up at Google as a developer, which give you access to some html code, which contains advertisements. Secondly, you inserts this piece of code at your website and asks Google to review it. If the use of the advertisement is approved their are presented at the website.  This is some time consuming steps and requires that the website would be finalized at a early stage of the project, which have not been the case. Therefore we have not implemented this at this stage of the project, but it would be, in a future version of the system. 

\textbf{Payment to uploaders}\\

As another part of the Business model in section \ref{BusinessModel} we states that if a user uploads a video, he or she will be paid for each rental or purchase of the video. This should contribute to the advantages of using our the RentIt system instead of others similar sites, as mentioned in section \ref{Strengths}, which describes the strengths of the system. The intended way to handle this was by using Paypal as outlined in section \ref{Paypal}. This would allow us to use a well known and secure source to handle transactions. But for the same reason as the previous section, with overhead of time, we decided to leave this subject for future development.

 




 