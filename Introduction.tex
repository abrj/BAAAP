\part{Introduction}
\section{Foreword}
There has been a great deal of development within the world of tutorials ever since the original idea of a tutor was first established at Oxford University in the 15th century. Back then, the scholars became responsible for the proper education of their fellow colleagues. Today, the expansive growth within the technological area, with the Internet as one of its central parts, has ensured a rapid development of on-line tutorials, making it possible for the average person to attain knowledge and know-how of virtually every imaginable subject, ranging from gardening to the creation of your own rocket. All that is needed is an internet connection along with a sense of curiosity and a willingness to learn. 

It is this quality of the Internet that we intend to take advantage of with this  project, which is part of the course Second Year Project With International Collaboration taught at the IT University of Copenhagen . The purpose of this project is to gain insight into the challenges of working with a geographically dispersed development team, under a number of constraints given by a formal project description, which we will present in the next section.

%Skal vi ikke udlade det her?
%Our site offers tutorials which is specifically centered around the area of technology and our longterm goal is to make it as close to self-sustainable as possible, by focusing on user generated content. One incentive for this is that uploaders will get paid for other users watching their videos, thus making sure that the tutorials are of as high quality as possible in order to get higher ratings and more users.%

\section{Project Description}
The project description is a document that contains a formal list of both goals and requirements which is a mandatory part of the course. It describes a system with a server which can store different types of media files and make them accessible for users, publishing the operations using a WCF interface.\\
In addition, developing a client application, which can be used to access the service, is a formal requirement.

Besides the requirements stated in the above, a central part of the project is an international collaboration. This is done with a group consisting of three students from Singapore Management University , which we will refer to as the SMU-group. We will further explain their background and role for this project in section \ref{InternationCollaboration}. \\

This concludes the description of the project and we will continue with a overview of the requirements.

\section{Vision}
In the following a system will be proposed, where it is possible for a user to rent, purchase, and upload different kind of media files. These media files will be in the form of tutorials and the authors of the files will be paid through the system when other users purchases or rents the media file. The service provides an overview of available media files for the users, which have been tagged by the authors. This gives the users the option to search for different files in categories and also rate the media files.\\

A user can either rent or purchase a tutorial. If the user rents a tutorial it will only be available for a given time period. If the user chooses to purchase a tutorial it will be available to the user for good. \\ 
In other words, our vision is to make a service which will include some educational technology related tutorials and  where the author of a good and thorough tutorial will be awarded financially for his work.

\subsection{Requirements} \label{Requirements}
The requirements from the project description can be outlined as list divided into project core requirements and additional features:

\subsubsection{Server}

\textbf{Core Requirements}
\begin{itemize}
	\item Must support one or more WCF services, written in C\#
	\item Must be deployed on an IIS 7 Server
	\item Must use a Microsoft SQL database
	\item The system will run in multi-user environments
	\item Users can register to create a new account, 
	\item Users can create, read, update and delete (CRUD) account information.
	\item Users must be able to login.
	\item Administrators can add or create new products
	\item Administrators can read, update and delete (CRUD). 		 
	\item The display of the product should provide information such as the thumbnail image, category, and 						analytical information (recent searches). 
	\item The system should support at least 2 media type.
	
	\item Users can search for a specific product.
	\item Users can rent and download products.
	\item The group can assume that there is already a player/viewer for the downloaded media.
	\item Most of the data should be stored on the server side.
\end{itemize}
\textbf{Nice-To-Have Requirements}
\begin{itemize}
	\item Based on a users previous rented or bought video files the server must be able to show recommendations
	\item Must give users the option to rate a tutorial
	\item Must be able to process and distribute payments between users and authors of the products
	\item Must support both administrator and normal user login, which gives different privileges
\end{itemize}

\subsubsection{Client}
\textbf{Core Requirements}
\begin{itemize}
	\item The UI for each of the features listed above. 
	\item The usability is important. For example:
	\begin{itemize}
		\item The login should hide the password.
		\item Homogeneous look and feel for all UI view
		\item Online help, tutorials, tooltips, etc.
	\end{itemize}
	\item The login should not repeat the password.
	\item CSS or template look and feel for each pages.
	\item Online help, tutorials, tooltips, etc.
\end{itemize}
\textbf{Additional Features}
\begin{itemize}
	\item A user must be able to log in using his Facebook credentials
	\item A user must be able to browse media files, sorted by rating
	\item The system must be able to make recommendations for users 
	\item A user must be able to rate a media file
	\item The system must provide shopping cart functionality
	\item A user must be able to see a history of his financial activity
	\item A user must be able to checkout his shopping cart and be granted access to the purchased media
	\item A user must be able to download a media file, which he has purchased. 
	\item A user must be able to upload media files to the system
	\item A user, with administrator privileges, must be able to manage users and media files 
\end{itemize}

\section{International Collaboration} \label{InternationCollaboration}
A big part of the project is to collaborate with a group from Singapore Management University. This SMU group is at a similar educational level as ourselves, but with a different technical background than us. The purpose of this collaboration is to give us experience similar to real-world situations, where it is common to collaborate with people across borders.\\
The two groups will have to agree on a specification for the server and decide what operations the server should have and what data types the client and server should exchange.
The SMU group will have to develop a client, which uses the server according to the specification the groups have agreed upon. The collaboration should be documented using a joint Wiki page, which all team members will have access to. A reflection of this collaboration will also be included in this report. Please refer to part \ref{SmuCollaboration} for further elaboration.

