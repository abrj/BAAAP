\part{Estimation}
\section{Overview}
\section{First techniques: Planing Poker and Delfi}
\subsection{Theory}

\subsection{Our experience}

\subsubsection{Pros}

\subsubsection{Cons}

\section{Second technique: PERT}
\subsection{Theory}

The Program (or Project) Evaluation and Review Technique, commonly known as PERT, is one of the more successful techniques in use in project planing today. The technique itself includes quite a lot of theory behind structuring tasks and their dependencies with each other, but also includes it own method of estimating.\\

The estimations in PERT relies on making multiple estimates for each task, and using that to calculate a weighted average. The estimates consists of a ``best case'', ``most likely'' and a ``worst case'' estimates. The weighted average is calculated is calculated as follows
$\frac{best+4*likely+best}{6}$


\subsection{Our experience}

\subsubsection{Pros}

\subsubsection{Cons}

\section{Third technique: ...}
\subsection{Theory}

\subsection{Our experience}

\subsubsection{Pros}

\subsubsection{Cons}

\section{Other techniques}
\subsection{CoCoMo}