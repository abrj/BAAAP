\part{Estimation}
\section{Overview}
Beskrivelse af de kommende ting og den generelle form for afsnittet for hver teknik
\section{First techniques: Delphi and Planing Poker}
In our first sprint we used a combination of the two techniques Delphi and Planning Poker.

\subsection{Theory}


The Delphi method is an estimation technique using the size of the collective experience and knowledge from the participating team members.

The technique is broken down into three main sections:
First, each participating team member is given a specification of the given work assignment which is to be carried out, and asked to give an estimate of the time it will take to finish it.
Second, the estimates from all participating team members are gathered and an average of these estimates are calculated. This average is shown to each of the team members.
Third, each team member have the opportunity to reconsider their own estimate relative to the average of the entire team, and, if they so choose, give their own revised estimate.
If a team member decides to change his or hers initial estimate, the whole process is run through once more. This is done until all team members are reasonably satisfied with the collective decision. 
The idea with the above mentioned steps are that since the individual estimates are kept private, there will be no incentive for the team members to begin a discussion about their individual disagreements. In situations like these, it can very often be the person who yells the loudest that gets his or her will, instead of the person who actually presents the most accurate estimate.


Planning Poker is similar to the Delphi technique on several points, including the fact that each team member gives a personal estimate for each individual task and also that it is possible to change your initial estimate later in the process. Where Planning Poker slightly differs from the Delphi method, is that you in Planning Poker get to see the estimate of each individual. If there is an estimation disagreement this can then spur some interesting discussions in order to find common ground. 

From our perspective, this can be conceived as both positive and negative. A positive side is, that you might be able to get a better idea of the actual time needed for the specific subject in question by looking at the estimate from persons who are experienced within the given subject. A negative side is, that there might be people in the team who are oncoming or arrogant in their behaviour, hence they might push the lesser confident people in their direction. 

\subsection{Our experience}


When using the Planning Poker technique, the fact that we could see each others estimates were helpful. There were situations where the estimates were in each end of the scale, and it created some pretty interesting discussions on why this was the case. By having these discussions we very often gained more insight regarding the specific subject, and most of the times each team member could contribute with info the others had not thought about or taken into consideration. It seemed like a good way to get a view from another perspective. We do strongly feel that an important factor here is that the team knows and trusts each other and also can speak freely among each other.



\subsubsection{Pros}

Delphi: 
- Only the average of all the estimates are shown which means that nobody will lose face in front of the others
- Each team member has the possibility to change their initial estimate after seeing the average estimate

Planning Poker:
- All estimates are shown which can give each team member a chance to explain their choice, potentially making it easier to come to an agreement. NB: this might very well only be on the positive side if the team members know each other and there does not exist any conflicts or large disagreements between team members, since this could create a narrow minded view from some team members (they only see things from their own perspective). Also, team members who shout loud to get their will can be a hindering for a good result.
- Each team member has the possibility to change their initial estimate after seeing the average estimate


Write about cost/benefit regarding time (it takes a lot of time, but often it gets outweighed by the fact that you get a clear specification of the individual tasks)


\subsubsection{Cons}

Delphi:
- 

Planning Poker:
- All estimates are shown. NB: this might only be a negative side if the team consists of any members that easily give in when in a discussion even though they do not agree with the others. This can especially be the case if there are some team members that shout loud to get their will. A situation like this can be hindering for a good estimate of the specific task at hand.

\section{Second technique: PERT}
\subsection{Theory}

The Program (or Project) Evaluation and Review Technique, commonly known as PERT, is one of the more successful techniques in use in project planing today. The technique itself includes quite a lot of theory behind structuring tasks and their dependencies with each other, but also includes it own method of estimating.\\

The estimations in PERT relies on making multiple estimates for each task, and using that to calculate a weighted average. The estimates consists of a ``best case'', a ``most likely'' and a ``worst case'' estimate. The weighted average is calculated as follows
$\frac{best+4*likely+worst}{6}$
, resulting in a number that is centred around the likely estimate, but can tilt to either sides.\
 
This estimate, or median, is a better guess for how long a task will take to fulfil. But there is still a deviation. We here talk about the standard deviation, which we calculate as such
$\frac{worst-best}{6}$
, and represents the ``tilt'' in the estimate. With this, we can interpret our median and the original three estimates as a probability, how likely a task is to finish within a time frame. This probability is not certain, but depends on different beta distributions used to express the spread of the likelihood !!REFERENCE!!. This is a bigger mathematical and statistic problem, which we will not discuss, but more can be found in the article !!REFERENCE!!.\

A general distribution for this is though used a many projects and gives a pretty good picture of how these estimates can be used.\
\begin{itemize}


\item The PERT estimate itself (the median) accounts for 50 \% of the probability.
\item Pert estimate +/- standard deviation = 84 \% probability.
\item Pert estimate +/- 2 x standard deviation = 97.5 \% probability.
\item Pert estimate +/- 3 x standard deviation = 99.5 \% probability.
\end{itemize}
An example would be that the three estimates of 3/5/10 hours would give you a median of 5.5, and a standard deviation of 1.17. So there is a 50 \% chance the task will be finished within 5.5 hours, but adding the standard deviation there is a 99.5 \% chance the task is finished in 1.9-9.1 hours.\
The PERT-estimate and deviations no longer directly reflects the original estimate, and have drifted somewhat, but these new numbers, derived from a probability, should end up being truer than what the team originally estimated. \\



\subsection{Our experience}
Utilizing this technique was a new experience for us, and changed the way we had formerly viewed estimation. The very scientific approach, utilizing mathematical analysis to achieve our results, help us to distance ourself from the different tasks, and sped up the process of estimation. The feeling that our guesses would not be the precise and final estimate, drove us to discuss less in terms of how much time we thought would be used for a task, and more in the lines of how uncertain we were about what would be needed for the task to be successfully finished within the deadline. \\

Beginning the sprint and starting implementation and completing tasks, we did though get a bit detached from the estimates, because of the very "floating" nature for the numbers, due to them being probabilities instead of actual estimates. This detachment combined with the fact that during this sprint a lot of unexpected work arose from our international collaboration, made the accuracy of these estimates hard to judge. \\

\subsubsection{Pros}
\begin{itemize}
\item The time it takes to make the estimates is reduced by calculating the deviations mathematically instead of basing it around a discussion.
\item Giving a variable and broad intervals of estimation helps to visualise the estimates as part of a process, not just some arbitrary number.
\item Having a deviation to go by makes it easier to quickly spot if a task is within an acceptable range from the median estimate, not having to judge it "by eye".

\end{itemize}

\subsubsection{Cons}
\begin{itemize}
\item Detachment from the estimates, seeing them as probabilities, makes it harder to follow them during the implementation process.
\item When making the estimates it can be tempting to not think each sub-estimate through, as it will be "hidden" behind the calculated estimate.

\end{itemize}

\section{Third technique: Direct estimation based on project breakdown}
\subsection{Theory}
The direct estimation technique can be used in situations where a project has been broken down into smaller tasks. After a detailed list of tasks has been created, one or several estimators will give their assessment on the effort needed to perform each task. The total effort needed for the entire project is then calculated by summarizing all the individual task estimates.
The technique is mainly used in the developing plans for sub-stages of a project, while other, for that level, more approximate methods are used until enough detailed information has been acquired. From here on out it is again possible to use the direct estimation technique. ****CHECK UP ON THAT LAST SENTENCE****

\subsection{Our experience}
Our experience with the direct estimation technique has been a little vague. We did not feel that it contributed with anything that was not already implemented in the previous tested methods or techniques.

\subsubsection{Pros}
- If the estimators possess a decent amount of knowledge and experience within the subjects of the tasks, very reliable estimates can be given. 

\subsubsection{Cons}
- This technique can be difficult to use in the beginning phase of a project, since here it is not always that all necessary information is available.
- The direct estimation technique can be very time consuming, and thereby costly, to use.
- If the estimators are inexperienced the time and resources used on estimation can possibly be wasted

\section{Other techniques}
\subsection{CoCoMo}



















