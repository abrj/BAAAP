\part{Estimation}
\section{Overview}
\section{First techniques: Planing Poker and Delfi}
\subsection{Theory}

\subsection{Our experience}

\subsubsection{Pros}

\subsubsection{Cons}

\section{Second technique: PERT}
\subsection{Theory}

The Program (or Project) Evaluation and Review Technique, commonly known as PERT, is one of the more successful techniques in use in project planing today. The technique itself includes quite a lot of theory behind structuring tasks and their dependencies with each other, but also includes it own method of estimating.\\

The estimations in PERT relies on making multiple estimates for each task, and using that to calculate a weighted average. The estimates consists of a ``best case'', a ``most likely'' and a ``worst case'' estimate. The weighted average is calculated as follows
$\frac{best+4*likely+worst}{6}$
, resulting in a number that is centred around the likely estimate, but can tilt to either sides.\
 
This estimate, or median, is a better guess for how long a task will take to fulfil. But there is still a deviation. We here talk about the standard deviation, which we calculate as such
$\frac{worst-best}{6}$
, and represents the ``tilt'' in the estimate. With this, we can interpret our median and the original three estimates as a probability, how likely a task is to finish within a time frame. This probability is not certain, but depends on different beta distributions used to express the spread of the likelihood !!REFERENCE!!. This is a bigger mathematical and statistic problem, which we will not discuss.\

A general distribution for this is though used a many projects and gives a pretty good picture of how these estimates can be used.\
\begin{itemize}


\item The PERT estimate itself (the median) accounts for 50 \% of the probability.
\item Pert estimate + standard deviation = 84 \% probability.
\item Pert estimate + 2 x standard deviation = 97.5 \% probability.
\item Pert estimate + 3 x standart deviation = 99.5 \% probability.
\end{itemize}
An example would be that...


\subsection{Our experience}

\subsubsection{Pros}

\subsubsection{Cons}

\section{Third technique: ...}
\subsection{Theory}

\subsection{Our experience}

\subsubsection{Pros}

\subsubsection{Cons}

\section{Other techniques}
\subsection{CoCoMo}