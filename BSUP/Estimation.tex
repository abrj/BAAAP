\part{Estimation}
\section{Overview}
\section{First techniques: Delphi and Planing Poker}
In our first sprint we used a combination of the two techniques Delphi and Planning Poker

\subsection{Theory}


The Delphi method is an estimation technique using the size of the collective experience and knowledge from the participating team members.

The technique is broken down into three main sections:
First, each participating team member is given a specification of the given work assignment which is to be carried out, and asked to give an estimate of the time it will take to finish it.
Second, the estimates from all participating team members are gathered and an average of these estimates are calculated. This average is shown to each of the team members.
Third, each team member have the opportunity to reconsider their own estimate relative to the average of the entire team, and, if they so choose, give their own revised estimate.
If a team member decides to change his or hers initial estimate, the whole process is run through once more. This is done until all team members are reasonably satisfied with the collective decision. 
The idea with the above mentioned steps are, that since the individual estimates are kept private, there will be no incentive for the team members to begin a discussion about their individual disagreements. In situations like these, it can very often be the person who yell the loudest that get his or hers will, instead of the person who actually presents the most accurate estimate.


Planning Poker is similar to the Delphi technique on several points, including the fact that each team member gives a personal estimate for each individual task and also that it is possible to change your initial estimate later in the process. Where Planning Poker slightly differs from the Delphi method, is that you in PLanning Poker get to see the estimate of each individual. If there is an estimation disagreement this can then spur some interesting discussions in order to find common ground. 

From our perspective, this can be conceived as both positive and negative. A positive side is, that you might be able to get a better idea of the actual time needed for the specific subject in question by looking at the estimate from persons who are experienced within the given subject. A negative side can be, that there might be people in the team who are oncoming or arrogant in their behaviour, hence they might (subconsciously?) push the lesser confident people in their direction. 

\subsection{Our experience}

Our experience with the Delphi technique and planning poker


When using the Planning Poker technique, we liked the fact that we could see each others estimates. There were situations where the estimates were in each end of the scale, and it created some pretty amusing discussions on why this was the case. By having these discussions we very often gained more insight regarding the specific subject, and most of the times each team member could contribute with info the others had not thought about. It seemed like a good way to get a view from another perspective. There were also situations where We do strongly feel that an important factor here is that the team knows and trusts each other and also can speak freely among each other.

\subsubsection{Pros}

Delphi: 
- Only the average of all the estimates are shown which mean that nobody will lose face in front of the others
- Each team member has the possibility to change their initial estimate after seeing the average estimate

Planning Poker:
- All estimates are shown (CAUTION: this might very well only be on the positive side if the team members know each other and there does not exist any conflicts or large disagreements between team members, since this could create a narrow minded view from some team members. Also, team members who shout loud to get their will can be a hindering for a good result).

\subsubsection{Cons}

Delphi:


Planning Poker:


\section{Second technique: PERT}
\subsection{Theory}

\subsection{Our experience}

\subsubsection{Pros}

\subsubsection{Cons}

\section{Third technique: ...}
\subsection{Theory}

\subsection{Our experience}

\subsubsection{Pros}

\subsubsection{Cons}

\section{Other techniques}
\subsection{CoCoMo}