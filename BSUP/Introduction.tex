\part{Introduction}
\section{Foreword}
For the course 'kursusnavn' have it been required to conduct an investigation for a certain subject and then reflect over this subject. We have choosen to focus our attention on estimation techniques for development in software projects. Based on readings (kilde?) and own experience, we think that this subject have been a hard thing to do in and have often been a  troublesome task to perform. Because of this, we have chosen to base our investigation on estimates and more precisely estimation techniques given from the course book 'kilde'. 

\section{Problem definition}
For a software project, where the team will consist of 5 people, how does a certain estimation technique impact the outcome of the project regarding to quality of the software as well as how the final deadline were handled. 
\begin{itemize}
\item What kind of estimation techniques is 'optimal' to use for a software project and how can these be incorporated in some of the earlier phases of the development
\item How does the techniques different from each other and what are the pros and cons for the various techniques
\item How can a team benefit by using a certain estimating technique
\item What and why does the change of success for a software project depend so deeply upon the estimation of the tasks?
\item 

\end{itemize}
\section{Motivation}
In the start of every software project, the first things that stakeholders, executive people and other similar persons are interested in, is the price of the project and the time frame that he project is about to enter. Questions like: "When can we expect the first prototype" and "When is the product ready to be shipped of to customers", are ones that often is needed to be answered in the start of a project typical by a manager. For the managers point of view, this can often be a challenge to answer, mainly due to the fact the 'requirements' for the project, at this point, are often vaguely defined and more a wish-list than actually requirements, that can be translated into program features. This combined with the fact, that developing a pierce of software from scratch often require a certain amount of innovation. Even though the project  will use known and common technologies and developing methods, these can be combined in a new way and this can both be time consuming and some time troublesome. \\
Because of this, many estimates for software project have been poorly and predicted delivery dates have often been exceeded with several percentages (KILDE OOA/D bogen). This could point towards that estimating projects should be handled by specialist which would have an objective view of the project, like its seen in civil engineering. Here an estimator does nothing else than trying to accurate estimate the tasks of the project. This means that estimates of civil engineering often are more precise than estimates in software projects. This is due to the fact, that in civil engineering when starting a project, its for the most time possible to use experience from earlier similar projects, which make the estimates more precise. \\
Clearly estimating in software projects is milestones behind civil engineering, where estimation specialist have a big importance among other things. Therefore a lot of research is done in the field of software engineering and many optimistic attempts have been made to develop estimating techniques, that tries to limit the chance of a estimations of tasks is so off, that it will impact the whole project and especially the final delivery date. 
Therefore we will, in the following sections, describe, discuss and analyse various estimating techniques and test some of them in a real software project. Many techniques have been developed and some have archived more success than others, but it is clear that we have to limit our scope and focus on only a couple of techniques. Because of this, the following sections will address and describe the following techniques: Delphi, planning poker, PERT and .... . \\

