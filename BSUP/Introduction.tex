\part{Introduction}
\section{Foreword}
For the course 'kursusnavn' have it been required to conduct an investigation for a certain subject and then reflect over this subject. We have choosen to focus our attention on estimation techniques for development in software projects. Based on readings (kilde?) and own experience, we think that this subject have been a hard thing to do in and have often been a  troublesome task to perform. Because of this, we have chosen to base our investigation on estimates and more precisely estimation techniques given from the course book 'kilde'. 

\section{Motivation}
In the start of every software project, the first things that stakeholders, executive people and other similar persons are interested in, is the price of the project and the time frame that he project is about to enter. Questions like: "When can we expect the first prototype" and "When is the product ready to be shipped of to customers", are ones that often is needed to be answered in the start of a project typical by a manager. For the managers point of view, this can often be a challenge to answer, mainly due to the fact the 'requirements' for the project, at this point, are often vaguely defined and more a wish-list than actually requirements, that can be translated into program features. This combined with the fact, that developing a pierce of software from scratch often require a certain amount of innovation. Even though the project  will use known and common technologies and developing methods, these can be combined in a new way and this can both be time consuming and some time troublesome. \\
Because of this, many estimates for software project have been poorly and predicted delivery dates have often been exceeded with several percentages (KILDE OOA/D bogen). This could point towards that estimating projects should be handled by specialist which would have an objective view of the project, like its seen in civil engineering. Here an estimator does nothing else than trying to accurate estimate the tasks of the project. This means that estimates of civil engineering often are more precise than estimates in software projects. This is due to the fact, that in civil engineering when starting a project, its for the most time possible to use experience from earlier similar projects, which make the estimates more precise. \\
Because of the things above we have decided to aim this report at task estimates for a software project. We will be following the SCRUM methodology and use various estimate technique from the book (BSUP BOG REFERENCE), to estimate the software project. This project will be a project 