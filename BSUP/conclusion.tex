\part{conclusion}
\section{Conclusion}
Estimation of software development projects is difficult. It requires a lot of prerequisites and experience to
be able to estimate a software project correctly. We have throughout this report tried to analyse some different
estimation techniques to try to find out if some techniques are more suitable than others for different kinds of
projects.\\

When it comes to the prerequisites the breakdown of tasks is one of the most important aspects, and one of the turning factors of how successful your estimation process will be. This is all based upon the ``goal'' of the software project, the requirements. These factors form the basis of good estimations.\\

A lot of factors and conditions also plays a role during both the estimating itself and the implementation process, where the estimates will play their role. Group-size, how intensively the project is run, and what kind of affiliations the members of the group have with each other, all lends themselves to different work processes, and therefore different estimation techniques. Outer factors like clients, collaborators and employers can all affect the effectiveness of the estimates, by providing challenges or information that change the conditions for the projects, both before but also during the project-run. 

Very structured projects, with even and steady work schedules, can benefit from deeper and more complex estimates, like the ones the PERT techniques supply, as these estimates gives better estimates but needs experience to utilize. These kind of estimates also lends themselves well to bigger projects, with bigger tasks, as the method of producing thorough probabilities for a whole project, adding the probabilities for the tasks together, can easier describe a larger system of features more precisely, but needs the very specific requirements specification.

On the other hand, smaller projects with a tighter project group, and less given requirements, benefits greatly from having an estimation technique that relies heavily on very precise breakdown of tasks, forcing the team to take decisions that is needed to describe the features of the system, where the requirements specification is lacking. The Planning Poker and Delfi methods prompt for this by trying to work your way toward the estimates, getting better inside in the process. Planning Poker explicitly with discussions, that brings the team to the same level. This helps secure an agreement of the estimates which, if the team is experienced enough, gives better precision.

These factors all affect how well the team can work with the tasks and estimates during the implementation, helping them to reach their deadlines. But an estimation technique that fits the working conditions also helps make these deadlines possible to reach, with more precise tasks and estimates.\\

It is clear that projects that resembles that of a 4 month university project, lends itself well to a technique such as Planning Poker, as it fits many of the conditions described above. But that said, as we stated in the start of the report, there is no one superior estimation technique. If the project we have tested the methods on was more compact, with a better specification, it could have utilized some of the benefits of PERT, or if our team had more data from earlier projects, we could have used some of the methods we have not described in this investigation. 

At the start of a project, the prerequisites and conditions should always be analysed, so a concious decision can be made towards picking the estimation technique that will fit the project the best


* What worked for us?
* Best results for 4 months project
* What could be better?
* How does the different techniques affect the teams' ability to estimate?

\subsection{Faults and errors}
We will now try to outline, some of the decisions we have made throughout the course of writing this report and running the parallel project, and how they can have effected or tainted the outcome of our conclusion.\\

One of the earliest decisions we made, was that we would apply the SCRUM method for structuring our work in the RentIt project. This greatly affected what kind of possibilities we had to work with in estimating our tasks. We have, as described, done estimating three times in the project, at different times and with different tasks. This made it hard to compare the estimation methods directly, and the changing conditions for the project, gave different and unpredictable effects on the different sprints. As a result our comparison of the techniques is lacking some measurable pointers for success. Utilizing a plan driven method would have given some structure that could have supported these problems better.

Not technically a decision of ours, the internationality of the RentIt project, presented unnecessary changes to the project along the way, further affecting the basis of our investigations. Trying to keep the project stable, as to measure the success of the estimations proved difficult.

We have also run this project along with other courses, and not being able to devote all our time to it, working at skewed times, not always all together. This made it hard to track exactly how much work was being done, lowering the precision of our measurements.


