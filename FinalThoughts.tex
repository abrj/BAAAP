\part{Final Thoughts}
In this part of the report we will present the reader with the final thoughts about the project and whether or not we have fulfilled all requirements from section \ref{Requirements} and the relation to the use-cases.

\section{Future Perspective}
For this section we will describe the features and functionality that we would like to have implemented in the system, but for different reasons didn't do.\\


\textbf{Advertisements}\\

As mentioned in section \ref{BusinessModel} the system would be financed through advertisements. The original idea was to start every rented video with a short commercial from Google Ad's. This would generate the needed financial base for being able to pay the uploaders of the tutorials and at the same time generate some profit for the system. We have not implemented this due to an extensive approval process at Google and the overhead of time it would take to implement it in a way that would be approved by Google. The approval requires two steps. The first is signing up at Google as a developer, which give you access to some html code, which contains advertisements. Secondly, you inserts this piece of code at your website and asks Google to review it. If the use of the advertisement is approved their are presented at the website.  This is some time consuming steps and requires that the website would be finalized at a early stage of the project, which have not been the case. Therefore we have not implemented this at this stage of the project, but it would be, in a future version of the system. 

\textbf{Payment to uploaders}\\

As another part of the Business model in section \ref{BusinessModel} we states that if a user uploads a video, he or she will be paid for each rental or purchase of the video. This should contribute to the advantages of using our the RentIt system instead of others similar sites, as mentioned in section \ref{Strengths}, which describes the strengths of the system. The intended way to handle this was by using Paypal as outlined in section \ref{Paypal}. This would allow us to use a well known and secure source to handle transactions. But for the same reason as the previous section, with overhead of time, we decided to leave this subject for future development.

\textbf{Exception Handling}
With every software system there is always a chance of errors and failure. This can both be user generated or errors caused by the server. These errors is handle with exceptions, through try and catch blocks. If an exceptions is raise due to an error, some of the exceptions should be displayed for the user, if it make sense, and the rest should be logged in the database, so a maintenance team would have the option to correct the errors. We have chosen to consistently use WebfaultExceptions, which are displayed to the user as a Http response status codes. This is the standard way of communicating with a a client in webservices and this is why we chose to do so. This also means that the server responses with a 2xx code when a request have been processed with success. This briefly explains our thoughts for the exceptions. The problem though is that in the WCF class, some of the catch blocks does not catch specific exceptions, but just exceptions in general. If the development team had more time at hand, this would be one thing to reconsider at reimplement, since it would make the system easier to debug in case of errors or faults and it would point towards a more stable system.  
 
\textbf{Tests}
In section \ref{TestingStrategy} we describe the thoughts about the importance of testing and how this should be done in this project. We have to some extend lived up to this and tested on the basis of our use-cases. This would ensure that the functionality the use-cases requires is working and tested. On the other side, we do however believe that since this is a system, without any complicated and complex datastructures or algorithms, we do not see it necessary to unit test every piece of code written. We have tested the system in a black-box kind of way, combined with relevant unit tests. We have made unit tests and made them perform an operation of the service and then looked in the database and file system, to see if the tables and files were updated and expected. We are aware of the pitfalls of doing this, instead of proper unit tests, but because of the relative simple and non-complex system, we feel that the method explained is enough to being able to say that the system performs as intended.

\textbf{Recommended Videos}
A system with the main function of providing large amounts of user-generated content, has to present different and easy access to this content. One of the ways we had thought of, was to implement some kind module for the RentIt service, that would provide the user with a list of videos that would fit the users preferences in some way. This feature is also present in our use-cases \ref{UseCases} and WFC interface!!REF!!, but was never implemented.

Our thoughts on the problem of a so called 'Recommender system', ranged from recommending videos based on statistics of what other users with the same age, gender, local etc. had watched or rated highly, known as a 'collaborate filtering' approach, or to recommend 'similar' videos to the ones you had already bought or rented, typically referred to as a 'content based' approach. Also, as we have incorporated Facebook integration into our user-profiling, we would be able to even further build on the 'collaborate filtering' by taking in to account what your Facebook-friends appreciates. \

The Wikipeadia article on 'Recommender systems'!!!!REF http://en.wikipedia.org/wiki/Recommender_system!!!! given a good inside on the topic, and in addition many articles can be found on the topic of de-constructing the algorithms of services such as Amazon, Youtube, Netflix etc. \

\textbf{Search functionality}
SHOULD WE JUST QUICKLY IMPLEMENT THIS?!?!\

\textbf{Encryption}
Our system handles sensitive information and credit card transactions, and as such security is an important issue, especially encryption of these data. We have not been able to focus on all parts of this system, and have therefore simplified it in terms of the security. This could though be an interesting extension of our system. More on the possibilities and decisions in the 'Security' section on page \ref{Security}. \





 