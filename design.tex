\part{Design}
In the previous sections we have discussed and analysed different solutions to the requirements presented in the first part, under section \ref{Requirements}. We will now proceed with describing the implementation of the system and outline the features that fulfils the requirements.

\section{Flow}
To let the user get a feel of the system at hand, we will in this section present different diagrams, with associated descriptions, which should ease the readers overall understanding of the system.
As with many others systems, the user has to log in to get full access. As mentioned, we have chosen to use Facebook services, to assist with the login process. It is possible for the user, to get different lists of videos, but for any other servicecall, the user must provide his access token. To obtain this token a call made to the 'LoginWithFacebook' operation has to made. When the server receives the request it is processed like shown below:

SD OF AUTHENTICATION

As seen, the request goes to several classes, before ending up calling Facebook via the address pointing to the blue oval. This request to Facebook ensures that the app does something....!
The next Facebook request to Facebook, is made to validate that the user has allowed the RentIt system app to access his data. If both requests responds with the correct information, the system proceeds with updating a map, containing all users that are logged in, with the newly user. This map is then used every time a user tries to call an operation, where validation is required. The WCF service then checks if the map already contains the given token and proceeds without futher requests to Facebook, if this is the case. Otherwise the validations proceeds as the above diagram shows.  


This RESTfull request to facebook uses the built-it class 'WebRequest', which takes an url to facebook as parameter and 
REFERE TILbAGE TIL FACEBOOK KRAV

\section{WCF operations and annotations}


\section{Security and Facebook}

