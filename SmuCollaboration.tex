\part{SMU Collaboration} \label{SmuCollaboration}
\section{Introduction}
For the Second Year Project: Software Development in Large Teams with International Collaboration we were divided into groups of the size of
4-6 persons. We were then assigned a group from Singapore Management University. The group that we were assigned had three members. Two boys and one girl.
When we were assigned to a group we started making the first connections. This lead to the two groups filling out some personal information about
the group members on the shared Wiki site. Further more we exchanged Skype informations and created a Facebook group for the day to day communication.

For the first meeting we prepared a little presentation of ourselves and had some thought about different forms of communication that we could use
throughout the project. We also selected two group members to be responsible for the daily communication with the SMU group. 
During the meeting we arranged the course of our future communication. This resulted in a mandatory Skype meeting every Tuesday at 10:00 danish time.
We also agreed of having extra meetings if either one of the groups felt that this was necessary. After this first meeting, both groups were now ready
to start working on the project.

\section{Communication}
In general both groups were good at attending the arranged meetings at the agreed time.
Both groups also used Facebook to communicate if there were any complications or questions they wanted answered or explanations they wanted further collaborated.
Especially our group did our best to explain the technical terms in as simple ways as possible to avoid too many misunderstandings. It did however lead to some
complications due to the fact that the technical knowledge of the two groups were quite different.
We experienced that the SMU group was not always very good at keeping us up to date on their status and problems. This could easily be a consequence of the cultural differences
that we had been made aware that exists between Singapore and Denmark. One of the things that were pointed out that we had to be very aware of, is the fact that a direct no
can be considered impolite in Singapore. This was a problem due to the fact that if we explained something to them, we could not always be sure that they had fully understood
what we actually meant, since they might had considered it impolite to answer 'no'. This showed especially in the last week before the SMU deadline where we suddenly found out
that there were some of the things that we had considered obvious that they had misunderstood. Another quite big misunderstanding that occurred, was that the SMU group wrote
to us five days before their deadline to make sure that we finished our part before the deadline. We had once before been made aware of this deadline, but had forgotten 
about it. We think that this once more shows the cultural difference between the two countries. In Denmark we would have written to the other group a lot of times
to make sure that they were on schedule and would be finished in time. Fortunately we solved the problem by working a couple of very long days in a row. This misunderstanding
could however easily have been avoided, had the Singaporeans been better at communicating. 

Another complication was the time difference. Often when we were coding in the afternoon and encountered problems or questions that were crucial to solve for us to continue,
we had to wait until the next day to get an answer, because it was night in Singapore and the SMU group was sleeping. Most of the time we were able to work on something else,
but it was quite frustrating, when the problem or question was crucial to our further work.
It is suggested (REFERENCE TO SLIDES) that it is a good idea to spend some of the initial time of an international collaboration to get to know the other team or teams.
Unfortunately this was not something that we prioritized very highly due to the fact that the SMU group had a deadline much earlier than we did, which meant we had to start
working on the project right away. We do however think that it would have been nice to know the SMU group on a personal level. This would have caused the two groups to feel
a bigger responsibility towards each other and thereby improve the collaboration. A better personal relationship could also have been obtained through our weekly planned
Skype meetings, but unfortunately most of the time, the Internet connection of the SMU group was so bad that they were unable to stream video or voice. This resulted in 
our group talking over Skype while they used the Skype chat function. This made the whole collaboration part of the project very impersonal, but even worse, did it also
result in some quite serious communication problems, since it was very difficult to have good discussions about anything.


* Barrier that we couldn't see their code and they weren't good at explaining their problems
* We were aware of communicating in a polite manner with them
* What did we or they fail to communicate?
* What could have done better from the beginning?
* Something about what we have learned is important for future international collaborations

