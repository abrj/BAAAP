\part{SMU Collaboration} \label{SmuCollaboration}
\section{Introduction}
For our Second Year Project: Software Development in Large Teams with International Collaboration we were divided into groups of the size of 4-6 persons. We were then assigned to a group from Singapore Management University. The group that we were assigned had three members, two boys and one girl.
When we were assigned to a group we started making the first connections. This lead to the two groups filling out some personal information about the group members on the shared Wiki site. Further more we exchanged Skype information and created a Facebook group for the day to day communication.

For the first meeting we prepared a little presentation of ourselves and had some thought about different forms of communication that we could use throughout the project. We also selected two group members to be responsible for the daily communication with the SMU group. 
During the meeting we arranged the course of our future communication. This resulted in a mandatory Skype meeting every Tuesday at 10:00 danish time.
We also agreed on having extra meetings if either one of the groups felt that this was necessary. After this first meeting, both groups were now ready to start working on the project.

\section{Collaboration}
In general, both groups were good at attending the arranged meetings at the agreed time.
Both groups also used Facebook to communicate if there were any complications or questions they wanted answered or explanations they needed to be further elaborated. It made good sense for us to use this media for ongoing questions because of the time difference. Each team could answer questions from the other when they had the time, and not only when both sides were available. However, the time difference did create some additional issues as well. Often, when we were coding in the afternoon and encountered problems or questions that were crucial to solve for us to continue, we had to wait until the next day to get an answer, because it was night in Singapore and the SMU group was sleeping. Most of the time we were able to work on something else, but it could be quite frustrating when the problem or question was crucial for our further work.
Regarding the specification of tasks, we especially did our best to explain the technical terms in as simple ways as possible to avoid too many misunderstandings. It did however lead to some complications due to the fact that the technical knowledge of the two groups were quite different, which were something we then needed to take into account.
We furthermore experienced that the SMU group was not always very good at keeping us up to date on their status and current issues. This could easily be a consequence of one the cultural differences that we had been made aware of, which exists between Singapore and Denmark. One of the things that were pointed out that we had to take extra notion of, was the fact that a direct no to an argument can be considered impolite in Singapore. This was a problem due to the fact that if we explained something to them, we could not always be sure that they had fully understood what we actually meant since they might considered no an impolite answer. This especially showed up in the last week before the SMU deadline where we suddenly found out that there were some of the things that we had considered obvious that they had misunderstood. Another quite big misunderstanding that occurred around the same time, was that the SMU group wrote to us five days before their deadline to make sure that we finished our part in due time. We had once before been briefly made aware of this deadline, but had forgotten about it since there had been no further emphasis on it. We think that this once more shows the cultural difference between the two countries. In Denmark we would have written to the other group a lot of times, preferably on a daily basis, to make sure that everything went according to the schedule and would be finished within the given time frame. Fortunately, we solved the problem by working a couple of very long days in a row. This misunderstanding could, however, easily have been avoided, had the Singaporeans communicated their status more often and precise or if we had had more knowledge about their way of communicating, thus being able to keep a close eye on their progress ourself. 
It is suggested (REFERENCE TO SLIDES) to spend some time during the initial part of an international collaboration to get to know the other team or teams on a more personal level. This is partly because of the fact that good relationships and networks should be established before you need to make use of them., which in our current case means we should put emphasis on this in the beginning of the collaboration. Another part is in order to build trust between the collaborating teams. Unfortunately, this was not something that we prioritized very high due to the fact that the SMU group had a deadline much earlier than we did, which meant we had to start working on the project right away. We do however think that it would have been nice to know the SMU group on a more personal level. This would have caused the two groups to feel a bigger responsibility towards each other and thereby further improving the collaboration. A better personal relationship could also have been obtained through our weekly planned Skype meetings, but unfortunately most of the time, the Internet connection of the SMU group was so bad that they were unable to stream video or voice. This resulted in our group talking over Skype while they used the Skype chat function. This made the whole collaboration part of the project very impersonal, but even worse, it also resulted in some quite serious communication problems, since it was very difficult to have good discussions about our subjects at hand.


\section{Theory}

According to Judith and Gary Olson, culture is something that is acquired and it helps people understand the world around them via habits and rules. This is why cultural differences can be of such a large magnitude.
 
According to Catherine Durnell and Pamela Hinds, teams who has an attitude that embraces the cultural and demographic differences are more likely to learn from the difficulties that emerges, hence evolving and expanding their cross-national collaboration abilities. By embracing and accepting the differences it is also possible to become a more creative team. This is partly because of the diversified points of view. On the other hand, the result of not embracing the differences can lead to more or bigger conflicts, such as ethnocentrism. Ethnocentrism is a factor that decreases the creativity of the collaboration between the teams, because a a specific team might want everything according to their own opinion, and also compares everything with this opinion.
Interdependence can make the team members more prone to engage with the others, and by fostering an attitude of positive distinctiveness is is more likely that the teams will avoid to nurture an ethnocentric culture. The same applies to equal status between team members. Both of the before mentioned parts will allegedly increase the cross-national learning.
Other factors that can have an impact on the collaboration can be traffic flow, opening and closing time of shops, and similar factors. We do not believe that this has played a crucial role in our project, since we laid our time schedule out early on and decided on a meeting time once a week which there was mutual agreement on. But we can imagine how it can be an important factor if the project was at a larger scale where a factor such as daily meetings might b essential for the project success.


\section{Lessons learned}


It is a good idea to have at least one fixed meeting day per week. This will ensure that we keep in contact throughout the duration of the project. Some projects will require more than one day per week, but the important part about the fixed days is that each individual team member can plan around them.

One of the most important lessons we learned and can take with us from this project, is the vital importance of upholding the communication between the teams at all times, even though no apparent issues are visible. By sharing information, even though it does not seem important, some unknown issues might be revealed. Informal communication is a key issue in order for this to be a success.

Barriers to informal communication can be overcome by establishing a good relationship early on in the process. This can be done by using the appropriate tools, such as Skype, Facebook, and email. Also, if travels, and thereby personal meetings, are a part of the project plan, these should be held as early in the process as possible, in order to establish the before mentioned relationship.

We should never assume anything. This includes situations where you believe that the other team understood what you just explained to them. A way to aid this can be to keep a log of the meetings where specifications from the meeting are noted down. It is then important to compare these specifications with the actual work produced, and also to revise and update them if changes occur.

We also feel that we should have placed a higher emphasis on milestones and responsibilities. This should be done in the initial phases of the project.

It is a good idea to use several media channels for the communication. Skype can create a more personal contact and serves as a good way to discuss specifications and requirements, as long as the connection is stable. Facebook can be used to the day to day communication and questions.

Reduced responsiveness must be expected when the temporal difference between the collaborating teams are very large. This goes for communication via Facebook as well as email.

Each team should take responsibility for the communication and not just assume that they can rely on the other team or that the other team will do a given task without specifically agreeing on this. This also applies if the task in question seems obvious.

It worked well for us with a centralized communication. We had two persons from our team who was the point of contact for the SMU team. This can make it easier to create a close relationship since the collaborating team only has to relate to a small number of individuals.

In our project it was mainly the SMU team that relied on us. If we had been as dependent on them as they were on us, late deliverables from either side could have crippled the other teams workflow and hence possibly delaying the project as a whole.





* Barrier that we couldn't see their code and they weren't good at explaining their problems
* We were aware of communicating in a polite manner with them
* What did we or they fail to communicate?
* What could have done better from the beginning?
* Something about what we have learned is important for future international collaborations

