\part{Analysis}
In this section, we will attempt to give the reader insight into the rationale behind the decisions made in the course of the project, by reflecting on challenges and surprises uncovered by analysing the problem area. In other words, the section will try to explain \textit{why} the system has been designed as it has and why the process has been managed the way it has.

The section is structured by the various techniques and artefacts used in conducting the analysis. The primary topics covered in this section are:
\begin{itemize}
\item Business modelling
\item Requirements analysis
\item Data modelling
\item Architectural analysis
\end{itemize}
We begin the section by investigating and discussing the conceptual business model of the system.

\section{Business modelling} \label{BusinessModel}
The following will present a business model, which will demonstrate the main ideas behind the our conceptualized system, the advantages and the possible threats.

Throughout the extreme growth of the usage of the Internet, a similar growth has been seen for user-uploaded content, which have explode on sites like www.youtube.com and www.facebook.com. Especially media files, like video and picture, has surfaced all over these sites, mainly due to the fore coming of smartphones, with picture and video capturing capabilities. The content of these media files is often self-exposing, where people capture themselves or friends doing all kinds of different things and at different events. But other media file content is also extremely popular to upload, like news broadcasting, sport events, tutorials, movies etc. Our business concept will focus on tutorials.

Tutorials have been widely exposed on Internet, where people are using the Internet increasingly to find solutions on how to do things themselves. Because of this, we have decided to develop a system, where a user can download and upload home made tutorials and get paid for providing these files for the system. The payments will be financed through advertisements on the tutorials or by user payment, when a user is buying a video tutorial. The users will also be able to search video tutorials by category and ratings.

\subsection{Target group}
The target group is expected to posses general IT knowledge and know how to use a computer at a basic level. The system will target users who wants to learn how to do things at home and by themselves and users that wants to learn about a special topic. The system will focus on tutorials of a specific subject, technology, and subjects comprised in this. This also means that the users will have to have some kind of interests which can be related to technology, why the system will only contain video tutorials of this sort. Since the system is focused on users with a certain interest, we do not exclude any particularly groups, as long as the users knows how to use a computer and internet browser.  

\subsection{Threats}
When developing a system like this, there will always be either threats or weaknesses, or both, related to the system. In this case, the most obvious threat, is the competition from other similar sites. Video tutorials is a big part of the Internet and therefore there already is a lot of websites and programs, that offers the same content to users. And on top of that, users will have to pay for videos without advertisements on this system, which similar sites and programs offers for free. Another threat is the fact that the video content for the users, will be user generated and for at start, this content will be very limited, until the system have been exposed and users starts to upload tutorials.

\subsection{Strengths} \label{Strengths}
This system has a lot of strengths associated with it and will have a great chance to greatly impact the target user group. One of the major strengths of the system, is that all tutorials will be gathered in one place. Often when searching for information on how to do stuff, its needed to look in several places and sites, and this is always time consuming and frustrating. For this system, all tutorials for all subjects for technology will be in one place and give the users easy access to many related tutorials. On top of this, if a user chooses to upload a video, that user will get a chance to earn money for every view and download of the tutorial. This will also encourage the users who wish to upload tutorials, to make the tutorial in a good quality and with a great content, since this will increase their chance of getting earning more money, which will benefit the system.\\

In the preceding text, have the system been explained and the major aspects outlined. The system will contain video tutorials regarding technology and every subject under this, which the users can either see with advertisements or buy without advertisements. The tutorials will be uploaded by users and these users will earn money for each time their tutorial is either viewed or bought.
The system have both strengths and threats, which have been discussed in the above. These will be taken into consideration when developing the final system and the threats.\\

In the next section we will consider the requirements listed in section \ref{Requirements} and use these to produce use-cases for the functional requirements for the system. 

\section{Requirements Analysis}
Generally speaking, requirements can be divided into two categories; functional and non-functional. The functional requirements for the RentIt server are expressed by use-cases, that each encapsulate some functionality the system must posses in order to be a useful solution. The non-functional requirements are captured by the factor-tabel.
  These artefacts will serve as the basis for the discussion of the requirements in this section.
\subsection{Use Cases}
As mentioned, the use-cases capture the functional requirements of the system. The list of use cases was composed by examining the business model and identifying user goals associated with uploading, viewing and buying videos, creating user accounts and so on.

When writing use cases for the RentIt system, we noted that they could be grouped into three categories of usage:
\begin{itemize}
\item \textbf{User management}\\
Use cases concerning user profile creation, editing user profiles and authorizing users
\item \textbf{Video management}\\
Use cases concerning downloading, viewing and rating videos
\item \textbf{Transaction management}\\
Use cases concerning transactions between account balances and buying more RentIt credits.
\end{itemize}
In other words, the use cases reveal at least three major areas of responsibility within the required functionality of the system. These responsibilities imply a large scale organization of namespaces, by way of the GRASP principles.

In addition, the use cases provided some intuition about the control flow of some of the major operations of the system. Take for example user case no. 18, which describes paying for a video:\\

18)
User A wishes to buy and rent some videos
\newline Precondition: The user is logged in to the site
\newline Precondition: The user has an active account with money deposited
\newline Postcondition: The chosen videos is available to the user
\begin{itemize}
	\item 1. The user navigates to a video list
	\item 2. The user finds the videos he wishes to buy
    \item 3. Presses either the buy or rent buttons, and the videos are added to his Shopping Cart
    \item 4. Checks out his Shopping Cart
    \item 5. The user is granted access to the videos
\newline Alternate flow:
    \item 3a. The user has changed his mind, and removes some videos from his Shopping Cart. Continued from 4.
\newline Exceptional flow:
    \item 4a. The user does not want to buy videos anyway, and clears his Shopping Cart. The use-case termintes.
\newline Exceptional flow:
    \item 4b. The user does not have sufficient credits to buy and rent these movies. The use-case terminates.
\end{itemize}

While short and simple, the pre- and postconditions as well as the steps involved in the use case, reveal a number responsibilities and changes in control flow to be considered. For example, the precondition indicates that each user has a balance associated with her account, to which deposits are made  in a separate use case (no. 15). Furthermore, the following responsibilities need to be delegated to appropriate branches of the system
\begin{itemize}
\item withdrawing money from the users account
\item granting users access to videos upon successful payment
\end{itemize}
This insight can be used as inspiration when designing the large scale organization of namespaces, and the control flow between these packages.

Moreover, the use cases proved an effective way of communicating ideas with the SMU group. By comparing our use cases with their use case model, it was possible to reconcile the two groups expectations to the semantic meaning of the different operations of the system. Take for example an earlier iteration of use case no. 7, dealing with purchasing videos:\\

\textbf{A user wants to see a video and downloads it to his computer}
\begin{itemize}
\item Precondition: User is logged in
\item Precondition: User has enough credits to pay the video
\item Postcondition: The video is saved on the users computer
\item Postcondition: The appropriate amount is subtracted from the users account
\begin{enumerate}
\item Available videos are presented to the user
\item The user selects the video he wants to see
\item The user is presented with options to rent and buy
\item The user selects buy
\item The user specifies where the file should be saved
\item The download begins
\end{enumerate}
\end{itemize}

This use case was utilized as the basis of a discussion between our own and the SMU group, about the semantic meaning of the term "buy", by the end of which it was decided that downloading a video constituted a buy, whereas renting would involve streaming the video, which is also expressed in the use case.

\subsection{Factor Table}
\label{FactorTable}
As opposed to the functional requirements, the non-functional requirements are not expressed by some functionality the system must offer, but rather some behaviour or quality the system must posses. These are captured by a factor table, as described in \textbf{Larman}. The list of factors the factor table consists of, was devised by using the FURPS+ mnemonic, and brainstorming on predictable requirements based on the business model. This facilitated a fruitful discussion on topics such as security, reliability and performance requirements of the system. The factor table can be found below:
\begin{center}
\includegraphics[scale=1.3]{FactorTable.png}
\end{center}
The factors are mainly defined by the business model, in that making changes to the business model, would change the non functional requirements of the system.  For example, the decision to include user generated content greatly impacts factors such as protection of video content in the system, as this in some sense the property of the user who uploaded it.

Some of the usefulness of the factor table is its recording of quality scenarios, because it encouraged the group to reflect on the quantifying the requirement, as to make it testable. Realistically, it would not be possible to satisfactory implement and test all of the factors, but for the sake of exercise, the factor table was composed as if it was meant to be used in  real world situation. As a consequence of the limited timeframe of the project, we decided to limit the factors that we would \textit{actually} emphasize to:
\begin{itemize}
\item accuracy of search results
\item documentation of web service interface
\item persistence of user data
\end{itemize}

These factors were consensually understood as the most crucial and/or interesting requirements to pursue.

\section{Data Modelling}
Another major part of our problem analysis was devising a model of the data, partly in order to construct and refine a database design for persisting the necessary data, and in part to serve for inspiration for C\# classes.
This modelling was done using two artefacts, a domain model to serve as a visual dictionary for objects and concepts in the problem domain, and an ER-diagram as an aid for reflecting on a useful database design. These two artefacts serve the basis for discussions in this section.
\subsection{Domain Model}
Something about the domain model
%The domain model was used to facilitate and document the results of a %discussion of the problem domain. By modelling the objects and concepts in the %problem domain, the project group came upon a number of questions, such as:
%\begin{itemize}
%\item 	
%\end{itemize}
\begin{center}
\includegraphics[scale=0.15]{DomainModel.png}
\end{center}

\subsection{ER-Diagram}
The ER diagram on the following page reflects our considerations on which data should be captured and persisted by the system. The video and user entities constitute the bare minimum of required data to implement the core functionality of video streaming and user management, and the remaining entities capture the data needed to record either deposits and payments, or the data needed to perform statistical analysis for e.g making recommendations to individual users.
\begin{center}
\includegraphics[scale=0.15]{ERDiagram.png}
\end{center}
The transaction, rented and bought tables together capture the financial activities of the users, in that transactions capture deposits, or payments by users for videos, and rented and bought capture the respective purchase histories of users. 
We later realised that this design is a flawed, as it allows for inconsistency between recorded purchases and financial transactions. A better design would have been to record a transaction with each rent- or buy history, and discard recording videos associated with a financial transaction directly in the transaction table.

Another noteworthy detail is the relationship between the price entity and the video entity. The reason to keep prices enumerated by category in a separate table, as opposed to recording it directly in the video table, is the business model requirement of allowing for user generated content. Defining a set of price categories that uploaders could choose from simplifies the upload process, as well as maintaining consistent pricing for different videos.

The shoppingcart entity was added to accommodate a wish from the SMU group of working with shopping cart functionality. Originally, the intent was for users to simply pay directly for each video separately when viewing it, either by renting or buying it. However, it was implemented since the SMU group was quite keen on the functionality, and it was very easily added to our existing model.

\section{Architectural Analysis}
In this section we will outline our efforts towards translating the functional and non functional requirements into a large scale organization of system namespaces, that is to say the logical architecture of the system. This is modelled using an UML package diagram, which will serve as the basis for the discussion.
\subsection{Package Diagram}
The package diagram below reflects the responsibilities identified by developing the use cases, and in part the non-functional requirements of the factor table. For example, the need for restricting access to videos dictates the existence of of modules handling authentication and authorization, which in the model is delegated to the UserManagement package. The level of abstraction used in the diagram is very high, and is intended as inspiration for classes and namespaces, for improving understanding of the system, but not as an exact specification.
\begin{center}
\includegraphics[scale=0.15]{PackageDiagram.png}
\end{center}
The WCF package assumes responsibility of exposing the service interface to clients, including which ever configuration is necessary for achieving this. Responsibility of adding items to, removing items from, emptying and checking out the shopping cart is delegated to the Finance package, in addition to making deposits to the users' RentIt credit balances.
The VideoManagement package is responsible for creating streams for the appropiate video files upon request, as well as generating links that can be used to download videos, in addition to limiting access to these links as to make them available only to paying customers. Finally, the Persistence package is responsible for saving data and coordinating requests to the underlying persistence technology.

A detailed discussion of possible technical approaches to realising the requirements within the proposed architecture is deferred to the next section.
\subsection{Technology Considerations}
One of the goals of conducting a thorough analysis of the requirements and architecture of the system under development, is to make an informed decision on how to best use existing technologies and frameworks. In this section we will give the reader an overview of our thoughts on this subject. The following sections are structured by the topics they address:
\begin{itemize}
\item Web Service standard
\item Client video streaming
\item Security
\item Finance
\item Persistence
\end{itemize}

\subsubsection{Web Service Standard}
A formal requirement for the project is that the server is implemented using the Microsoft WCF framework. Consequential, the server operations will be exposed to clients as a service. The WCF framework offers a wide variety of possible standards to let clients communicate with it. The major decision in this regard is choosing which message format is most appropriate; SOAP or REST.

choosing the SOAP standard has the advantage of being the WCF standard, which means using this option is likely to result in less configuration of the framework, thus saving the group some time and possibly grief. In addition, SOAP works equally well over a multitude of different networking protocols, since it does not depend on the HTTP vocabulary, in contrast to REST, which provides some flexibility. This adds little value to the project however, as there is no good reason to use anything other than the HTTP protocol for communicating with clients.

Disadvantages of choosing SOAP over REST include the verbosity of the XML notation of SOAP messages, which cause them to be larger than the basic HTTP messages of REST. Moreover, the choice of service standard affects the ease with which different clients will communicate with the service. For example, choosing REST which returns and accepts with JSON datatypes and therefore readily communicates with AJAX based clients, would make using  HTML5 video streaming quite easy. Using SOAP on the other hand would require a few extra contortions to extract the video information from the SOAP envelope, and play it using the HTML5 video element. On the other hand, .NET and Visual Studio have tools for working with the WSDL (Web Services Description Language) of SOAP services, in order to automatically serialize and de-serialize C\# data to SOAP envelopes. For this reason, streaming video clients written in C\#, such as WPF or Silverlight, should be straightforward to implement.

Since the SMU group expressed preference to writing an ASP/AJAX client, there were strong arguments for supplying a REST service in order to make their lives easier. For implementing our own client, our own group is partial to HTML5 video aswell (for reasons which will be discussed in the next section), REST seemed like the better choice. in the end, using the REST standard to implement the service was chosen for its compatibility with HTML/AJAX.

\subsubsection{Client Video Streaming}
Another formal requirement for the project, was to implement a client for the RentIt service of our own. Firstly, this begs the question of whether to offer a desktop or web client. Offering a web client would allow it to be used on any platform that supports the web standards, which includes tablet and mobile devices. These devices could be supported by using a framework for adapting the user interface for the screen size of the specific device, like for example Bootstrap. Implementing desktop clients however, is much more familiar to the group, and could possibly save us some time for working on other features for system. Most importantly, the group deemed the learning 	outcome of working with web technologies to be greater of the two. Web technologies arguably play an increasing larger role in the landscape of IT products, and as such, familiarising ourselves with state of the art web standards would be well worth the effort. For this reason, the group chose to implement a web client rather than a desktop.

After deciding on a web solution for the client part of the project, it was necessary to select the most appropriate technology for streaming video. The three major video streaming technologies, which can be considered web standards are Flash, HTML5 and Silverlight.
Flash video streaming was off the table from the start, becouse it does not posses the quality of being a state of the art web standard, since it arguably has largely played out its role on the web. Working with Flash was not educationally interesting.

HTML, being the de facto standard of everything on the web, will probably always be relevant to be familiar with as a software developer. The multimedia capabilities of HTML5, are increasingly being used by product providers, and is supported by all browsers on all platforms. However, using HTML5 for streaming video would make it difficult to prevent users from downloading the videos their own machines, for viewing even when the rent period had been exceeded, since HTML5 video treats the video elements exactly like any other resource, like a picture. This means that saving a video would only require the user to right-click the video element and select save as. Moreover, using HTML5 would make it cumbersome to use SOAP to communicate with the RentIt service, as dicussed above, since the streamed data would have to be extracted from the SOAP envelopes and placed somewhere and in some format the HTML element could point at.

Silverlight suffers from some of the same diseases as Flash, in that Microsoft has abandoned the technology. There are a few commercial products that still depend on Silverlight, such as Netflix, but technologies that are not maintained by their providers are bound to disappear. As such the educational outcome of using is Silverlight is negligible. The technology does have the advantage however, of being part of the Microsoft technology stack, even though it is now considered legacy. This means that writing parts of the client in Silverlight could take advantage of the tools for working with SOAP services with WSDLs, which is the default service standard WCF.

The HTML5 solution was chosen as the technology for handing video streaming for the following reasons:
\begin{itemize}
\item More interesting from an educational point of view
\item More broadly supported by browsers and mobile platforms
\end{itemize}

\subsubsection{Security} \label{Security}
The business model depends on the system being secure in a number of different regards. Firstly, the system must provide some means of offering authentication and authorization of users, in order to make sure that users will only be granted access to content once they have paid for it, and in the appropriate time frame. In addition, it should not be possible for a user to download a video, unless she has permanently bought it.

WCF in itself provides tools for handling authentication and authorization, based on the ASP.NET authentication framework!!REFERENCE TIL LEARNING WCF KAPITEL 7. Using that, it would be possible to implement a fairly generic login architecture. However, in the past few years, a different approach to authenticating users have become quite popular; namely using web identity providers. By this approach, users login with an account registered with a well known and widely used foreign service, such as Facebook, Google or Windows Live. This is convenient for users since they then need to remember less usernames and passwords. From an educational perspective, it would also be interesting because it would allow the project group to get hands on experience with service oriented architecture, since these identity providers work by providing a service API. In addition, the default WCF identity mechanisms require credentials to be formatted in a specific manner, and placed in the authorization header of the HTTP message. Using e.g Facebook authentication, involves passing around a unique string, which can be used to access a users' information!!REFERENCE TIL FACEBOOK DOKUMENTATION. This gives us a little more freedom for making simplifications to the login architecture. For those reasons we decided to focus on using identity providers, rather than using the default WCF authentication framework.

As mentioned, preventing users from using the streamed data obtained by renting a video to download a video to their own hard-drive is also an important security issue. This is a difficult problem to solve since when a user is streaming the video data in their browser, regardless of which technology is used, they already have access to the data. This is exemplified by the myriad of rippers that exist for other media solutions such as Youtube, Spotify, Soundcloud or Netflix to name a few. For this reason, it seems unlikely that we would solve this problem in this project, since our focus was with managing a project with a geographically dispersed team. As such, although we are aware of the issue, we have deemed it outside of the scope of this project.

Since sensitive information is passed by the client to the service, such as login and credit card information, implementing a secure communication channel between two is an important security issue as well. An obvious approach to this would be to use SSL encryption to establish a HTTPS connection, since this can all be handled automatically by the WCF framework and the IIS hosting environment!!REFERENCE TIL LEARNING WCF KAPITEL 7 OG MSDN. However, configuring the server on which  the IIS environment is running requires rights that are not granted to students, so this would have to be done by the course lecturer, in addition to all subsequent changes to the certificate. For this reason, we have decided to delimit to project to not include encryption.

\subsubsection{Finance} \label{Paypal}
Users pay for videos using a credit card. This means that the credit card needs to be validated by some external system, in order to ensure that the card is valid. Again we have decided to take advantage of service oriented architecture, by using PayPal as a merchant service provider. PayPal has a sandboxed service endpoint which makes testing easy, without using an actual credit card.

\subsubsection{Persistence}
Microsoft SQL Server is installed on the server, which is the most obvious choice for persisting data. However, it might be worth considering using a Object Relational Mapping framework to translate from C\# objects to relational tables, in order to save writing a substantial number of lines of boilerplate code. The ORM framework included in .NET is called Entity framework, which is fairly well integrated with Visual Studio: tools include GUI editing of table mappings, discovering so called EntityObjects through connecting to a database, just to name a few. In addition, the Entity Framework works very well with the .NET LINQ construct, in that its possible to query so called EntitySets with LINQ queries, which is then translated to SQL.

Another possibility is a framework called NHibernate, which offers much of the same functionality, but is not as tightly integrated with Visual Studio. Because of our familiarity with the Entity Framework, we decided to use the Entity Framework, in order to reduce development time on the persistence module of the system.

\section{Project Management Plan}
To manage the development phase of our project, we have utilized different methods and tools, trying to optimize our work flow and organizing the work that needed to be done. \

We have applied the SCRUM method during our implementations. As an agile development technique, we have chosen this, to try and be more flexible in our work, especially considering our international collaboration. As we expected some adjustments to our requirement had to be made in accordance with the SMU team, and to some degree SCRUM help handle these. But there were times when the opposite were true, and the changes to our work were to great. A more in-depth look at our work with agile methods and dispersed collaborations can be read in the SMU Collaboration section on page \pageref{SMU Collaboration}

The breakdown and estimations of task



SCRUM, Team Foundation
